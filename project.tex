% TODO:
% * include changes in the explanation text

\documentclass[12pt]{article}
\usepackage{amsmath}
\usepackage{amsfonts}
\usepackage[
    backend=biber,
    style=alphabetic,
    style=numeric,
    citestyle=numeric,
]{biblatex}
\usepackage[margin=5em]{geometry}
\usepackage[utf8]{inputenc}
\usepackage{listings}
\usepackage{lscape}
\usepackage{mathpartir}
\usepackage{stmaryrd}
\usepackage{syntax}
\usepackage{textcomp}

\addbibresource{bibliography.bib}

\lstset{
    frame=single,
    gobble=4,
    language=Python,
}

\setlength\parindent{0em}
\setlength\parskip{1em}

\newcommand\mono\texttt
\newcommand{\metavar}[1]{\textlangle#1\textrangle}
\newcommand{\dblbr}[1]{\llbracket#1\rrbracket}
\newcommand{\fancybr}[2]{#1 \dblbr{#2}}
\renewcommand{\AA}{\mathcal{A}}
\newcommand{\Ntwo}{Nielson \& Nielson}
\newcommand{\BB}{\mathcal{B}}
\newcommand{\CC}{\mathcal{C}}
\newcommand{\LL}{\mathcal{L}}
\newcommand{\NN}{\mathcal{N}}
\newcommand{\Var}{\mathbf{Var}}
\newcommand{\Avar}{\mathbf{Avar}}
\newcommand{\Lvar}{\mathbf{Lvar}}
\newcommand{\Aexp}{\mathbf{Aexp}}
\newcommand{\Bexp}{\mathbf{Bexp}}
\newcommand{\Lexp}{\mathbf{Lexp}}
\newcommand{\State}{\mathbf{State}}
\newcommand{\Astate}{\mathbf{Astate}}
\newcommand{\Lstate}{\mathbf{Lstate}}
\newcommand{\Llit}{\mathbf{Llit}}
\newcommand{\Lcomp}{\mathbf{Lcomp}}

\title{
    Python Comprehensions in While:\\
    Second Version
}
\author{
    Hendrik Werner s4549775
    \and Constantin Blach s4329872
    \and Jona Heidsick s0710512
}

\begin{document}

\maketitle

\begin{abstract}
\noindent % so there is no random space in the beginning of the abstract
In this document we extend the language While with list comprehensions inspired by Python. We explore the syntactical and semantic changes and additions needed to achieve this. Finally we analyze the extensions we made using a case study of a While program.
\end{abstract}

\tableofcontents

\section{Introduction}
Comprehensions are extremely useful for dealing with container types and functional constructs in imperative languages. Python is an example of an imperative language that uses comprehensions, which can be used as more idiomatic versions of \mono{filter} and \mono{map}. They are not limited to this use case though.

By combining and nesting list comprehensions, it is possible to implement complicated calculations with minimal syntax and maximum elegance. Take the following Python program as an example:

\begin{lstlisting}
    def cartesian_product(a, b):
        return [(x, y) for x in a for y in b]
\end{lstlisting}

It defines the Cartesian product in terms of list comprehensions. We want to be able to express this in the language While as well.

We want to use Natural Semantics (NS) to describe list comprehensions in While, because as opposed to Structural Operational Semantics (SOS) it offers a higher level overview over the semantics. SOS is concerned with the individual steps of a computation, while NS describes the overall results \cite{wiley}. The high level overview is a better fit to describe list comprehensions. Python is a great example of that. Python 2 and Python 3 both offer list comprehensions with the same semantics, but with a different implementation behind the scenes. This is not really relevant to the semantics though:

\begin{lstlisting}[title=Python 2/3]
    >>> [x * 2 for x in [1, 2, 3]]
    [2, 4, 6]
\end{lstlisting}

The individual steps are different for Python 2 and Python 3, but the overall result is the same.

Our goal is to eventually show the execution of the following While program:

\begin{lstlisting}
    x := 1:2:3:[];
    y := [a * 2 for a in x if a * 2 < 5]
\end{lstlisting}

The execution of this program should result in the state $s_{1:2:3:[], 2:4:[]}$, given the initial state $s_{\perp, \perp}$.

\section{First Principles}

\subsection{Syntactical extensions}

The syntax we need is given here in Backus-Naur form (BNF). The semantics with applications book uses something very similar to BNF, but it is not quite BNF \cite[section 1.2]{wiley}. We repeat the definitions given in the book here, using BNF, even if they did not change, for convenience.

\subsubsection{Metavariables}

\begin{itemize}
    \item \metavar{n} will range over numerals, \textbf{Num},
    \item \metavar{x} will range over arithmetic variables, $\Avar$,
    \item \metavar{y} will range over list variables, $\Lvar$,
    \item \metavar{a} will range over arithmetic expressions, \textbf{Aexp},
    \item \metavar{b} will range over boolean expressions, $\Bexp$,
    \item \metavar{l} will range over list expressions, $\Lexp = \Llit \cup \Lcomp$,
    \item \metavar{ll} will range over list literals, $\Llit$,
    \item \metavar{lc} will range over list comprehensions, $\Lcomp$, and
    \item \metavar{S} will range over statements, \textbf{Stm}.
\end{itemize}

\subsubsection{Syntax}

\begin{grammar}
    <a> ::= <n>
    \alt <x>
    \alt <a1> "+" <a2>
    \alt <a1> "*" <a2>
    \alt <a1> "-" <a2>

    <b> ::= "true"
    \alt "false"
    \alt <a1> "=" <a2>
    \alt $\lnot$<b>
    \alt <b1> $\land$ <b2>

    <l> ::= <ll>
    \alt <lc>

    <ll> ::= "[]"
    \alt <n>":"<ll>

    <lc> ::= "["<a> for <x> "in" <l> [if <b>]"]"
    \alt "["<a> "for" <x> "in" <y> [if <b>]"]"

    <S> ::= <x> ":=" <a>
    \alt "skip"
    \alt <S1> ";" <S2>
    \alt "if" <b> "then" <S1> "else" <S2>
    \alt "while" <b> "do" <S>
    \alt <y> ":=" <l>
\end{grammar}

\subsubsection{Differences}

In addition to the variables that are used by \Ntwo \cite{wiley}, we split the While variables into arithmetic variables and list variables. This allows us to prevent strange construction like

\begin{lstlisting}
    x := 4;
    y := 1:2:[z * 2 for z in x]
\end{lstlisting}

on a semantic level, so we don't have to deal with them "at run time". This is also convenient for the programmer, as a potential While compiler/interpreter could catch these errors early, instead of bailing out when trying to run the program.

\mono{x} is an arithmetic variable, which is not allowed to the position it does in the above code. This position needs to be filled by either a list \metavar{l} or a list variable \metavar{y}.

Additionally the definition of list variable \mono{y} begins with a list literal \metavar{ll}, and then continues with a list comprehension \metavar{lc}. The syntax we chose forbids this, as you have to choose between both versions when defining a list \metavar{l}.

The split of arithmetic variables \metavar{x} and list variables \metavar{y} makes While comprehensions a little less versatile, but arguably also less confusing than Python. As an example, take the following valid Python program, which makes a copy of the \mono{x} list:

\begin{lstlisting}
    x = [1, 2, 3]
    y = [x for x in x]
\end{lstlisting}

After executing the above program, \mono{x} and \mono{y} contain the same values. A naive translation into While would look like this:

\begin{lstlisting}
    x := 1:2:3:[];
    y := [x for x in x]
\end{lstlisting}

The above While program is not valid according to our syntax, because \mono{x} takes the position of a list variable and an arithmetic variable at the same time, which is forbidden. This is arguably for the better, as the Python program could be confusing because \mono{x} is used as a list and list element in the same expression.

\subsection{Semantic extensions}

\subsubsection{Semantic Functions}

\Ntwo define the state function as being of type $\Var \rightarrow \mathbb{Z}$, so we need to redefine it. We have two types of variables: $\Avar$ and $\Lvar$, so we also have two state functions.

\begin{itemize}
    \item $\Astate: \Avar \rightarrow \mathbb{Z}$

    This is just the function $\State$ from \Ntwo \cite{wiley}, but renamed to fit to our case with split list variables and arithmetic variables.
    \item $\Lstate: \Lvar \rightarrow \mathbb{N} \rightarrow \mathbb{Z}$

    This state function is used to hold the state of lists, which we represent as functions from the natural numbers to integers, which is a good fit to represent sequences of values, so for example:

    After executing the While program \mono{x := 1:2:[]}, $\Lstate(\mono{x}) = f[0 \mapsto 1, 1 \mapsto 2]$, so that $f(0) = 1, f(1) = 2, f(x > 1) = \perp$
\end{itemize}

The intersection of domains of $\Astate$ and $\Lstate$ is empty, such that $Astate(x) \neq \perp \rightarrow Lstate(x) = \perp \land Lstate(x) \neq \perp \rightarrow \Astate(x) = \perp$. This is because we want to use every variable name for \textbf{either} an arithmetic variable, \textbf{or} a list variable, so that we never end up with a list variable with the same name as an arithmetic variable. This way we can safely define $\State = \Astate \cup \Lstate$.

We also introduced a new function $\LL$, which translates between list literal expressions \metavar{ll} and the function representation we chose that defines the list.

$\LL: \Llit \rightarrow \mathbb{N} \rightarrow \mathbb{Z}$

$\begin{aligned}
    f(x) &= \perp & \fancybr{\LL}{h \mono{:} t, n} &= \fancybr{\LL}{t, n + 1}[n \mapsto \fancybr{\NN}{h}]\\
    \fancybr{\LL}{l} &= \fancybr{\LL}{l, 0} & \fancybr{\LL}{\mono{[]}, n} &= f\\
\end{aligned}$

The semantics of $\AA$, $\BB$, and $\NN$ are identical to the version in the book.

There are list comprehensions \metavar{lc} as well, which need to be converted to list literals \metavar{ll}. The converter function $\CC$ takes care of that.

$\CC: \Lcomp \rightarrow \State \rightarrow \Llit$

$\begin{aligned}
    \fancybr{\CC}{\mono{[} a \mono{ for } x \mono{ in } ll \mono{]}} s &= \fancybr{\CC}{\mono{[} a \mono{ for } x \mono{ in } ll \mono{ if true]}} s\\
    \fancybr{\CC}{\mono{[} a \mono{ for } x \mono{ in [] if } b \mono{]}} s &= \mono{[]}\\
    \fancybr{\CC}{\mono{[} a \mono{ for } x \mono{ in } h \mono{:} t \mono{ if } b \mono{]}} s &= \begin{cases}
        \fancybr{\AA}{a} s[x \mapsto h] \mono{:} \fancybr{\CC}{\mono{[} a \mono{ for } x \mono{ in } t \mono{ if } b\mono{]}} s
        & \text{if } \fancybr{\BB}{b} s[x \mapsto \fancybr{\NN}{h}] = \mathbf{tt}\\
        \fancybr{\CC}{\mono{[} a \mono{ for } x \mono{ in } t \mono{ if } b \mono{]}} s
        & \text{if } \fancybr{\BB}{b} s[x \mapsto \fancybr{\NN}{h}] = \mathbf{ff}\\
    \end{cases}\\
\end{aligned}$

In addition to that, we need an extra replace function $R$, which handles list variables inside our list comprehensions.
We cannot solve the following program with our current rules:

\begin{lstlisting}
    l := 1:2:[];
    y := [x for x in l]
\end{lstlisting}

To fix this, we add the function $R$ which replaces the list variables \metavar{y} in a list comprehension \metavar{lc} back into list literals \metavar{ll}, such that they can be evaluated by the $\CC$ function.
This $R$ function works as follows:

$R: \Lcomp \text{(with Lvar)} \rightarrow \Lcomp \text{(without Lvar)}$

$\begin{aligned}
    \fancybr{R}{\mono{[} a \mono{ for } x \mono{ in } y\mono{]}} &= \fancybr{R}{\mono{[} a \mono{ for } x \mono{ in } y \mono{ if true]}}\\
    \fancybr{R}{\mono{[} a \mono{ for } x \mono{ in } y \mono{ if } b \mono{]}} &= \fancybr{R}{\mono{[} a \mono{ for } x \mono{ in } \Lstate(y) \mono{ if } b\mono{]}}\\
\end{aligned}$

\subsubsection{Semantics}

We are using the same basic form as \Ntwo \cite[Section 2.1]{wiley} for our semantic rules.
The basic rule looks as follow:

\begin{mathpar}
    \inferrule*[right={if \dots}]
        {<S_1, s_0> \rightarrow s_1, \dots, <S_n, s_{n-1}> \rightarrow s_n}
        {<S, s_0> \rightarrow s_n}\\
\end{mathpar}

Rules are comprised of transitions, we also have a basic form for the transitions.
The basic transition is of the following form:

\begin{mathpar}
    \inferrule*[]
        {}
        {<S, s> \rightarrow s'}\\
\end{mathpar}

These basic rules allow the following semantic rules for our extended While language:

\begin{center}
\begin{tabular}{lll}
    $[\text{ASS}_a]$ & $\inferrule
        {}
        {<x \mono{ := } a, s> \rightarrow s[x \mapsto \fancybr{\AA}{a}s]}
    $\\\\
    $[\text{SKIP}]$ & $\inferrule
        {}
        {<\mono{skip}, s> \rightarrow s}
    $\\\\
    $[\text{COMP}]$ & $\inferrule
        {<S_1, s> \rightarrow s', <S_2, s'> \rightarrow s''}
        {<S_1 \mono{; } S_2, s> \rightarrow s''}
    $\\\\
    $[\text{IF}^{tt}]$ & $\inferrule
        {<S_1, s> \rightarrow s'}
        {<\mono{if } b \mono{ then } S_1 \mono{ else } S_2, s> \rightarrow s'}
    $ & if $\fancybr{\BB}{b}s = \mathbf{tt}$\\\\
    $[\text{IF}^{ff}]$ & $\inferrule
        {<S_2, s> \rightarrow s'}
        {<\mono{if } b \mono{ then } S_1 \mono{ else } S_2, s> \rightarrow s'}
    $ & if $\fancybr{\BB}{b}s = \mathbf{ff}$\\\\
    $[\text{WHILE}^{tt}]$ & $\inferrule
        {<S, s> \rightarrow s', <\mono{while } b \mono{ do } S, s'> \rightarrow s''}
        {<\mono{while } b \mono{ do } S, s> \rightarrow s''}
    $ & if $\fancybr{\BB}{b}s = \mathbf{tt}$\\\\
    % does this fit the general form? This is taken form the book...
    $[\text{WHILE}^{ff}]$ & $\inferrule
        {}
        {<\mono{while } b \mono{ do } S, s> \rightarrow s}
    $ & if $\fancybr{\BB}{b}s = \mathbf{ff}$\\\\
\end{tabular}
\end{center}

We reused the above mentioned rules from \Ntwo \cite[Table 2.1]{wiley}.
We deleted the "natural semantics" ($_{ns}$) part of each rule's name, as all of our rules are in natural semantics. Additionally, we changed the name of the assignment rule $[\text{ASS}]$ to $[\text{ASS}_a]$, as we are using an additional assignment rule for list literals \metavar{ll} and list comprehensions \metavar{lc} respectively.

\begin{center}
    $[\text{ASS}_{ll}]\ \inferrule
        {}
        {<y \mono{ := } ll, s> \rightarrow s[y \mapsto \fancybr{\LL}{ll}]}$
\end{center}

This rule allows assignments of list literals \metavar{ll} to a list variable y $\in \Lvar$.
After the execution of this assignment in state s, y is equal to $\fancybr{\LL}{ll}$.

\begin{center}
    $[\text{ASS}_{lc}]\ \inferrule
        {<y \mono{ := } \fancybr{\CC}{lc}s, s> \rightarrow s'}
        {<y \mono{ := } lc, s> \rightarrow s'}$
\end{center}

This rule makes it possible to assign a list comprehension to a variable.
To execute this rule, the $\CC$ function is used, which "translates" the list comprehension to a list literal.
After this translation, the assignment rule $[\text{ass}_{ll}]$ is used to assign the list literal to the variable y.

\begin{center}
    $[\text{LCOMP}_{var}]\ \inferrule
        {<y \mono{ := } \fancybr{R}{lc}, s> \rightarrow s'}
        {<y \mono{ := } lc, s> \rightarrow s'}$
\end{center}

We use this rule to replace the list variable inside the list comprehension with an actual list literal, which can then be used by the $\CC$ function to compute the resulting list literal from the list comprehension.
The R function takes a list comprehension with an $\Lvar$ as input and returns a list comprehension with an $\Llit$.

\section{Application}

$\equiv$ is used as in $\lambda$ calculus, meaning that the left hand side and right hand side are syntactically equivalent. We just give names to program snippets to make things clearer.

\subsection{List literal}
$s_x$ means that variable \mono{a} has the value $x$ in state $s$.

\begin{mathpar}
    \inferrule*[right={$[\text{ass}_{ll}]$}]
        {}
        {<\mono{a := []}, s_\perp> \rightarrow s_{[]}}
\end{mathpar}

\subsection{List comprehension with list literal}

\begin{mathpar}
    \inferrule*[right={$[\text{ass}_{lc}]$}]
        {\inferrule*[right={$[\text{ass}_{ll}]$}]
            {}
            {<\mono{y := 2:3:[]}, s_\perp> \rightarrow s_{2:3:[]}}
        }
        {<\mono{y := [x + 1 for x in 1:2:[]]}, s_\perp> \rightarrow s_{2:3:[]}}
\end{mathpar}

Because

$\begin{aligned}
    &\fancybr{\CC}{\mono{[x + 1 for x in 1:2:[]]}} s_\perp\\
    =\ & \fancybr{\CC}{\mono{[x + 1 for x in 1:2:[] if true]}} s_\perp\\
    =\ & \fancybr{\AA}{\mono{x + 1}} s_\perp[\mono{x} \mapsto \fancybr{\NN}{1}] : \fancybr{\CC}{[\mono{x + 1 for x in 2:[] if true}]} s_\perp
    & \text{because } \fancybr{\BB}{\mono{true}} s_\perp[\mono{x} \mapsto \fancybr{\NN}{\mono{1}}] = \mathbf{tt}\\
    =\ & \mono{2:} \fancybr{\CC}{\mono{[x + 1 for x in 2:[] if true]}} s_\perp\\
    =\ & \mono{2:} \fancybr{\AA}{\mono{x + 1}} s_\perp[\mono{x} \mapsto \fancybr{\NN}{\mono{2}}] : \fancybr{\CC}{[\mono{x + 1 for x in [] if true}]} s_\perp
    & \text{because } \fancybr{\BB}{\mono{true}} s_\perp[\mono{x} \mapsto \fancybr{\NN}{\mono{2}}] = \mathbf{tt}\\
    =\ & \mono{2:3:} \fancybr{\CC}{\mono{[x + 1 for x in [] if true]}} s_\perp\\
    =\ & \mono{2:3:[]}\\
\end{aligned}$

\subsection{List comprehension with list variable}

\begin{itemize}
    \item $S_1 \equiv \mono{l := 1:2:[]}$
    \item $S_2 \equiv \mono{y := [x + 1 for x in l]}$
    \item $S_2' \equiv \mono{y := [x + 1 for x in 1:2:[]]}$
    \item $\begin{aligned}
        s_0(\mono{l}) &= \perp & s_0(\mono{y}) &= \perp\\
        s_1(\mono{l}) &= \mono{1:2:[]} & s_1(\mono{y}) &= \perp\\
        s_2(\mono{l}) &= \mono{1:2:[]} & s_2(\mono{y}) &= \mono{2:3:[]}\\
    \end{aligned}$
\end{itemize}

\begin{mathpar}
    \inferrule*[right={$[\text{comp}]$}]
        {\inferrule*[right={$[\text{ass}_{ll}]$}]
            {}
            {<S_1, s_0> \rightarrow s_1}
        \inferrule*[right={$[\text{lcomp}_{var}]$}]
            {\inferrule*[right={$[\text{ass}_{lc}]$}]
                {\inferrule*[right={$[\text{ass}_{ll}]$}]
                    {}
                    {<\mono{y := 2:3:[]}, s_1> \rightarrow s_2}
                }
                {<S_2', s_1> \rightarrow s_2}
            }
            {<S_2, s_1> \rightarrow s_2}
        }
        {<S_1 \mono{; } S_2, s_0> \rightarrow s_2}
\end{mathpar}

Because

$\begin{aligned}
    & \fancybr{\CC}{\mono{[x + 1 for x in 1:2:[]]}} s_1\\
    =\ & \fancybr{\CC}{\mono{[x + 1 for x in 1:2:[] if true]}} s_1\\
    =\ & \fancybr{\AA}{\mono{x + 1}} s_1[\mono{x} \mapsto \fancybr{\NN}{\mono{1}}] \mono{:} \fancybr{\CC}{\mono{[x + 1 for x in 2:[] if true]}} s_1
    & \text{because } \fancybr{\BB}{\mono{true}} s_1[\mono{x} \mapsto \fancybr{\NN}{\mono{1}}] = \mathbf{tt}\\
    =\ & \mono{2:} \fancybr{\CC}{\mono{[x + 1 for x in 2:[] if true]}} s_1\\
    =\ & \mono{2:} \fancybr{\AA}{\mono{x + 1}} s_1[\mono{x} \mapsto \fancybr{\NN}{\mono{2}}] \mono{:} \fancybr{\CC}{\mono{[x + 1 for x in [] if true]}} s_1
    & \text{because } \fancybr{\BB}{\mono{true}} s_1[\mono{x} \mapsto \fancybr{\NN}{\mono{2}}] = \mathbf{tt}\\
    =\ & \mono{2:3:} \fancybr{\CC}{\mono{[x + 1 for x in [] if true]}} s_1\\
    =\ & \mono{2:3:[]}\\
\end{aligned}$

\subsection{Case study}

We have shown that the syntax and semantics we defined can be used to show the execution of simple programs. Now we can assemble the parts and tackle the case study of the program from the introduction.

\begin{itemize}
    \item $S_1 \equiv \mono{x := 1:2:3:[]}$
    \item $S_2 \equiv \mono{y := [a * 2 for a in x if } b \mono{]}$
    \item $S_2' \equiv \mono{y := [a * 2 for a in 1:2:3:[] if } b \mono{]}$
    \item $b \equiv \mono{a * 2 < 5}$
    \item $\begin{aligned}
        s_0(\text{x}) &= \perp & s_0(\text{y}) &= \perp\\
        s_1(\text{x}) &= \mono{1:2:3:[]} & s_1(\text{y}) &= \perp\\
        s_2(\text{x}) &= \mono{1:2:3:[]} & s_2(\text{y}) &= \mono{2:4:[]}\\
    \end{aligned}$
\end{itemize}

\begin{mathpar}
    \inferrule*[right={$[\text{comp}]$}]
        {\inferrule*[right={$[\text{ass}_{ll}]$}]
            {}
            {<S_1, s_0> \rightarrow s_1}
        \inferrule*[right={$[\text{lcomp}_{var}]$}]
            {\inferrule*[right={$[\text{ass}_{lc}]$}]
                {\inferrule*[right={$[\text{ass}_{ll}]$}]
                    {}
                    {<\mono{y := 2:4:[]}, s_1> \rightarrow s_2}
                }
                {<S_2', s_1> \rightarrow s_2}
            }
            {<S_2, s_1> \rightarrow s_2}
        }
        {<S_1 \mono{;} S_2, s_0> \rightarrow s_2}
\end{mathpar}

Because

$\begin{aligned}
    & \fancybr{\CC}{\mono{[a * 2 for a in 1:2:3:[] if } b \mono{]}} s_1\\
    =\ & \fancybr{\AA}{\mono{a * 2}} s_1[\mono{a} \mapsto \fancybr{\NN}{\mono{1}}] \mono{:} \fancybr{\CC}{\mono{[a * 2 for a in 2:3:[] if } b \mono{]}} s_1
    & \text{because } \fancybr{\BB}{b} s_1[\mono{a} \mapsto \fancybr{\NN}{\mono{1}}] = \mathbf{tt}\\
    =\ & \mono{2:} \fancybr{\CC}{\mono{[a * 2 for a in 2:3:[] if } b \mono{]}} s_1\\
    =\ & \mono{2:} \fancybr{\AA}{\mono{a * 2}} s_1[\mono{a} \mapsto \fancybr{\NN}{\mono{2}}] \mono{:} \fancybr{\CC}{\mono{[a * 2 for a in 3:[] if } b \mono{]}} s_1
    & \text{because } \fancybr{\BB}{b} s_1[\mono{a} \mapsto \fancybr{\NN}{\mono{2}}] = \mathbf{tt}\\
    =\ & \mono{2:4:} \fancybr{\CC}{\mono{[a * 2 for a in 3:[] if } b \mono{]}} s_1\\
    =\ & \mono{2:4:} \fancybr{\CC}{\mono{[a * 2 for a in [] if } b \mono{]}} s_1
    & \text{because } \fancybr{\BB}{b} s_1[\mono{a} \mapsto \fancybr{\NN}{\mono{3}}] = \mathbf{ff}\\
    =\ & \mono{2:4:[]}\\
\end{aligned}$

\section{Conclusion}

We have successfully demonstrated the viability of our semantic and syntactical extensions to the While language. They allow us to execute arbitrary While programs containing simple list comprehensions. We exemplified this by a case study on the program from the introduction. The derivation uses all semantic rules we introduced, and successfully terminates in the expected state.

\printbibliography

\end{document}
